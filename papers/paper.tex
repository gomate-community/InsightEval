\documentclass[sigconf]{acmart}
%% NOTE that a single column version is required for 
%% submission and peer review. This can be done by changing
%% the \doucmentclass[...]{acmart} in this template to 
%% \documentclass[manuscript,screen]{acmart}
%% 
%% To ensure 100% compatibility, please check the white list of
%% approved LaTeX packages to be used with the Master Article Template at
%% https://www.acm.org/publications/taps/whitelist-of-latex-packages 
%% before creating your document. The white list page provides 
%% information on how to submit additional LaTeX packages for 
%% review and adoption.
%% Fonts used in the template cannot be substituted; margin 
%% adjustments are not allowed.

%%
\input{preamble/packages}
\input{preamble/definitions}
% Zaiyu Xia, Qiang Yan, Yixing Fan, Jiafeng Guo, Ruqing Zhang and Xueqi Cheng
\author{
Zaiyu Xia$^{\dagger}$, 
Qiang Yan$^{\dagger}$, 
Yixing Fan$^{\dagger, \ddag}$ \orcidlink{0000-0003-4317-2702},
Jiafeng Guo$^{\dagger, \ddag}$, 
Ruqing Zhang$^{\dagger, \ddag}$ 
and Xueqi Cheng$^{\dagger, \ddag}$
}
\affiliation{
  \institution{
  $^{\dagger}$Key Lab of Network Data Science and Technology, Institute of Computing Technology,\\ Chinese Academy of Sciences, Beijing, China\\
  ${\ddag}$University of Chinese Academy of Sciences, Beijing, China\\
  } 
}
\email{{xiazaiyu24z, yanqiang, fanyixing, guojiafeng, zhangruqing, cxq}@ict.ac.cn}

\renewcommand{\shortauthors}{Xia et al.}

\setcopyright{rightsretained}
\copyrightyear{2026}
\acmYear{2026}
\acmDOI{XXXXXXX.XXXXXXX}

%% These commands are for a PROCEEDINGS abstract or paper.
\acmConference[SIGIR '26]{Proceedings of the 49th International ACM SIGIR Conference on Research and Development in Information Retrieval}{July 13--17, 2026}{Madrid, Spain}
%%
%%  Uncomment \acmBooktitle if the title of the proceedings is different
%%  from ``Proceedings of ...''!
%%
%%\acmBooktitle{Woodstock '18: ACM Symposium on Neural Gaze Detection,
%%  June 03--05, 2018, Woodstock, NY}
\acmISBN{978-1-4503-XXXX-X/26/07}


%%
%% end of the preamble, start of the body of the document source.
\begin{document}

%%
%% The "title" command has an optional parameter,
%% allowing the author to define a "short title" to be used in page headers.
\title{InsightEval: An Automated System for Evaluating Insightfulness in Scientific Papers}

%% The abstract is a short summary of the work to be presented in the article.
%% Abstract must be exactly as submitted
\begin{abstract}
In scientific writing, impactful work is distinguished by its ability to provide insight, which encompasses both what new information it introduces and the level at which it advances understanding beyond prior work. However, insight is inherently relative to existing literature and is rarely captured by evaluations that focus on writing quality or contribution overlap, leaving insight assessment largely underexplored. We present InsightEval, an automated system for evaluating the insightfulness of scientific papers by jointly assessing information gain and the level of understanding reflected in their claims, relative to cited references. Our approach is inspired by a cognitive hypothesis: after thoroughly reading all references cited by a paper, the extent to which reading the paper enhances a reader's understanding reflects its level of insight. The system operates through four stages: (1) extracting opinion sentences from the target paper; (2) retrieving supporting materials from cited references for each opinion sentence via semantic retrieval; (3) scoring each opinion sentence by jointly evaluating its information gain and its level of understanding—characterized along depth, breadth, and height—using a large language model conditioned on the retrieved supporting materials; and (4) synthesizing the sentence-level evaluations into a paper-level insight report. We deploy InsightEval on over 100 human-written and 100 AI-generated scientific papers, demonstrating its applicability across diverse writing sources. For transparency and inspection, we publicly release the InsightEval source code, a demonstration video, and all generated insight evaluation reports.
\end{abstract}

%%
%% The code below is generated by the tool at http://dl.acm.org/ccs.cfm.
%% Please copy and paste the code instead of the example below.
%%
% \begin{CCSXML}
% <ccs2012>
%  <concept>
%   <concept_id>00000000.0000000.0000000</concept_id>
%   <concept_desc>Do Not Use This Code, Generate the Correct Terms for Your Paper</concept_desc>
%   <concept_significance>500</concept_significance>
%  </concept>
% </ccs2012>
% \end{CCSXML}

% \ccsdesc[500]{Do Not Use This Code~Generate the Correct Terms for Your Paper}

%%
%% Keywords. The author(s) should pick words that accurately describe
%% the work being presented. Separate the keywords with commas.
%% Keywords must be exactly as submitted
\keywords{Insightfulness, Evaluation, Deep Research}



%%
%% This command processes the author and affiliation and title
%% information and builds the first part of the formatted document.
\maketitle

\section{Introduction}
% 作者分工:夏再禹

The rapid proliferation of scientific publications has made it increasingly challenging for researchers to identify papers that provide genuine intellectual contributions beyond surface-level reporting~\cite{bornmann2015growth}. While existing tools effectively measure citation impact~\cite{garfield1972citation}, novelty detection~\cite{wang2017novelty,ai-etal-2025-novascore,zhang2026opennoveltyllmpoweredagenticverifiable,Arts_2025}, or writing quality~\cite{ke2019automated}, a critical dimension remains largely unexplored: \textit{insightfulness}---the degree to which a paper advances understanding beyond the knowledge contained in its references.

We define insightfulness as the \textit{argumentative gain} that an author achieves when synthesizing, critiquing, or extending prior work. Consider two scenarios: (1) an author who simply restates findings from reference papers, and (2) an author who identifies gaps, synthesizes disparate findings, or proposes novel interpretations. The latter demonstrates higher insightfulness, yet current evaluation systems---including automated peer review frameworks~\cite{zhu-etal-2025-deepreview,taechoyotin2025remorautomatedpeerreview}, AI-driven scientific writing tools~\cite{lu2024aiscientistfullyautomated,zheng-etal-2025-automation}, and novelty assessment methods~\cite{shahid-etal-2025-literature,Guo2024IdeaBenchBL}---cannot distinguish between these cases.

Inspired by a cognitive hypothesis---that the value of reading a paper can be measured by the extent to which it enhances understanding \textit{beyond} what could be learned from reading all its references alone---we present \textbf{InsightEval}, an automated system for evaluating the insightfulness of scientific papers. InsightEval operationalizes insightfulness through three measurable dimensions:
\begin{itemize}
    \item \textbf{Depth}: Does the author provide deeper analysis or more fundamental understanding of a concept?
    \item \textbf{Breadth}: Does the author synthesize findings across multiple references to reveal broader patterns?
    \item \textbf{Height}: Does the author abstract findings to a higher conceptual level or propose generalizable principles?
\end{itemize}

\begin{figure}[!t]
\centering
% TODO: Insert system architecture diagram
\fbox{\parbox{0.95\columnwidth}{\centering [Figure Placeholder: System Architecture Diagram]\\ Showing the four-stage pipeline: Opinion Extraction $\rightarrow$ Evidence Retrieval $\rightarrow$ Insight Scoring $\rightarrow$ Report Synthesis}}
\caption{An Overview of the InsightEval System Architecture.}
\label{fig:architecture}
\end{figure}

Our system addresses several key requirements of the SIGIR Demonstrations Track. \textit{Target users} include senior reviewers seeking efficient paper quality assessment, junior researchers learning to write insightful introductions, and research institutions conducting systematic evaluation of scholarly output. \textit{The problem} we address is the gap between quantitative bibliometrics and qualitative assessment of intellectual contribution~\cite{Arnaout2025IndepthRI,Vo2024AssessingSI}. Unlike citation analysis tools (e.g., Connected Papers~\cite{connectedpapers}) that focus on network structure, or critical reading assistants (e.g., InsightGUIDE~\cite{Koloveas2025InsightGUIDEAO}) that provide opinion-level guidance, InsightEval examines the \textit{semantic relationship} between a paper's claims and its supporting references and quantifies insightfulness along interpretable dimensions.

In this demo, we introduce InsightEval, a novel system to evaluate the insightfulness of scientific papers in a comprehensive way. The overall architecture of the system consists of two main components:
1) \textbf{the InsightEval Library}: an easy to use evaluation library which implements the four-stage evaluation pipeline, including opinion extraction, evidence retrieval, insight scoring, and report synthesis;
2) \textbf{the InsightEval Studio}: a user-friendly and interactive Web interface which enables users to upload papers, configure evaluation parameters, and inspect detailed insight reports.
Our work makes the following key contributions:
\begin{itemize}
    \item \textbf{The InsightEval Studio for Interactive Paper Evaluation}: The studio enables users to evaluate papers without writing any code. It features an \textit{augmented document reader} with color-coded insight highlighting and an \textit{insight analysis panel} that displays evidence alignment, radar charts, and AI rationale for each opinion sentence.
    \item \textbf{The InsightEval Library for Multi-Dimensional Insight Scoring}: The library provides a comprehensive pipeline covering opinion extraction, semantic retrieval from cited references, and LLM-based scoring along three interpretable dimensions (depth, breadth, height), allowing researchers to build upon and customize the evaluation framework.
    \item \textbf{Empirical Deployment on Human-Written and AI-Generated Papers}: We deploy InsightEval on over 100 human-written and 100 AI-generated scientific papers, demonstrating its applicability across diverse writing sources and revealing distinctive patterns in insightfulness.
    \item \textbf{An Open-Source Implementation}: We open-source the InsightEval framework and provide comprehensive documentation. The InsightEval studio and library are publicly accessible at \url{https://github.com/gomate-community/2026-SIGIR-InsightEval}.
\end{itemize}

\section{Related Work}
\label{sec:related}

Our work intersects with three lines of research: LLM-based automated peer review, automated scientific writing and literature review, and novelty and insight assessment for scholarly documents. We briefly survey each area and highlight how InsightEval differs from and complements existing approaches.

\subsection{LLM-based Automated Peer Review}

The growing volume of scientific submissions has motivated the development of automated peer review systems powered by large language models. DeepReview~\cite{zhu-etal-2025-deepreview} proposes a multi-stage framework that emulates expert reviewers through structured analysis, literature retrieval, and evidence-based argumentation. REMOR~\cite{taechoyotin2025remorautomatedpeerreview} combines LLM reasoning with multi-objective reinforcement learning to generate comprehensive reviews. Beyond generation quality, researchers have investigated the reliability and robustness of LLM-based reviewers: Lin et al.~\cite{lin-etal-2025-breaking} reveal significant vulnerabilities in LLM reviewers under textual adversarial attacks, while Farber~\cite{Farber2025ComparingHA} presents a comparative analysis of human and AI expertise in the peer review process. Wu et al.~\cite{Wu2024AreTC} further examine whether reviewer confidence scores are consistent with review content in top AI conferences. At the meta-evaluation level, Goldberg et al.~\cite{goldberg2024usefulnessllmsauthorchecklist} investigate the usefulness of LLMs as author checklist assistants at NeurIPS.

While these systems focus on generating holistic quality assessments of papers (e.g., soundness, clarity, and significance), they do not specifically evaluate the \textit{insightfulness} of a paper's claims relative to its cited references. InsightEval addresses this gap by providing a fine-grained, evidence-grounded evaluation of the intellectual contribution a paper makes beyond its references.

\subsection{Automated Scientific Writing and Literature Review}

A parallel line of work leverages LLMs for automating scientific writing and literature synthesis. Lu et al.~\cite{lu2024aiscientistfullyautomated} propose the AI Scientist, a system for fully automated open-ended scientific discovery, which has been evaluated by Beel et al.~\cite{beel2025evaluating} for its practical viability. Zheng et al.~\cite{zheng-etal-2025-automation} provide a comprehensive survey on LLMs in scientific discovery, charting the evolution from task-specific automation to increasingly autonomous research agents. For structured long-form generation, STORM~\cite{shao-etal-2024-assisting} introduces a multi-perspective question-asking approach for writing Wikipedia-like articles grounded in retrieved sources. SurveyGen-I~\cite{chen-etal-2025-surveygen} addresses the challenge of consistent survey generation through evolving plans and memory-guided writing. Wu et al.~\cite{Wu_2025} present an automated literature research and review generation method, while Zhao et al.~\cite{zhao2025literaturereviewliteraturereviews} offer a meta-level literature review of survey papers in pattern analysis and machine intelligence. VERIRAG~\cite{mohole2025veriragpostretrievalauditingscientific} tackles the post-retrieval auditing of scientific study summaries to ensure factual accuracy.

These systems primarily focus on the \textit{generation} and \textit{organization} of scientific text but do not assess the quality of insights produced. InsightEval complements these tools by providing a principled framework to evaluate whether generated or human-written content achieves genuine intellectual advancement beyond its source materials.

\subsection{Novelty and Insight Assessment}

The most closely related line of work concerns automated assessment of novelty and insight in scholarly documents. Arts et al.~\cite{Arts_2025} pioneer the measurement of novel scientific ideas directly from publication text, moving beyond citation-based metrics. NovAScore~\cite{ai-etal-2025-novascore} proposes an automated metric for evaluating document-level novelty by aggregating novelty and salience scores of atomic information units. OpenNovelty~\cite{zhang2026opennoveltyllmpoweredagenticverifiable} introduces an LLM-powered agentic system for verifiable scholarly novelty assessment through literature-grounded comparison. Shahid et al.~\cite{shahid-etal-2025-literature} further develop a literature-grounded framework for assessing the novelty of scientific ideas using retrieval-augmented generation with faceted re-ranking. IdeaBench~\cite{Guo2024IdeaBenchBL} provides a benchmark for evaluating LLMs' ability to generate novel research ideas. For impact-oriented analysis, Arnaout et al.~\cite{Arnaout2025IndepthRI} propose fine-grained temporal citation analysis for in-depth research impact summarization. InsightGUIDE~\cite{Koloveas2025InsightGUIDEAO} offers an opinionated AI assistant for guided critical reading of scientific literature, while Vo and Simmie~\cite{Vo2024AssessingSI} provide a systematic review of tasks, tools, and techniques for assessing scientific inquiry.

While novelty assessment evaluates the \textit{newness} of ideas relative to prior work, it does not measure the \textit{depth of understanding} a paper achieves. A paper may introduce no novel methods yet still provide deep, integrative insight by synthesizing, reinterpreting, or abstracting known findings. InsightEval distinguishes itself by jointly evaluating information gain and the level of understanding---characterized through depth, breadth, and height---thereby capturing a complementary dimension that existing novelty metrics do not address.

\section{Evaluation Framework}
\label{sec:framework}
% 作者分工:闫强

The InsightEval library implements a four-stage evaluation pipeline. Each stage is modular and can be independently configured or extended.

\subsection{Stage 1: Opinion Sentence Extraction}
The first stage processes the input PDF document to identify opinion sentences---statements that reflect the author's viewpoints, interpretations, or claims.

\heading{PDF Parsing} We utilize MinerU~\cite{mineru} to extract structured content from academic PDFs, preserving section boundaries and inline citations. The parser identifies citation markers (e.g., ``[1]'', ``Smith et al.'') and associates them with the corresponding sentences.

\heading{Sentence Classification} We categorize sentences into three types:
\begin{itemize}
    \item \textbf{Context sentences}: Background statements without citations or author opinions.
    \item \textbf{Citation sentences}: Sentences containing references but merely summarizing prior work.
    \item \textbf{Opinion sentences}: Sentences expressing the author's interpretation, critique, or synthesis, often marked by discourse markers (e.g., ``However'', ``We argue that'', ``Crucially'').
\end{itemize}
A fine-tuned classifier based on \ac{LLM} identifies opinion sentences with high precision for subsequent evaluation.

\subsection{Stage 2: Evidence Retrieval via RAG}
For each opinion sentence, we retrieve relevant supporting materials from the cited references using a \ac{RAG} approach.

\heading{Reference Resolution} Citation markers are resolved to their corresponding reference entries. For each cited paper, we retrieve the abstract, introduction, and conclusion sections using the Semantic Scholar API~\cite{semanticscholar} or a local document repository.

\heading{Semantic Retrieval} We encode the opinion sentence using a dense retriever (e.g., Sentence-BERT~\cite{reimers2019sentence}) and retrieve the most semantically similar passages from each cited reference. This produces evidence pairs $\{(o_i, E_i)\}$, where $o_i$ is an opinion sentence and $E_i$ is the set of retrieved evidence passages.

\subsection{Stage 3: Multi-Dimensional Insight Scoring}
This stage constitutes the core contribution of InsightEval. For each $(o_i, E_i)$ pair, we evaluate the insightfulness along three dimensions using an LLM-based scoring framework.

\heading{Scoring Dimensions} We design prompts for an \ac{LLM} (e.g., GPT-4~\cite{openai2023gpt4} or DeepSeek-V3~\cite{deepseek2024}) to assess:

\begin{enumerate}
    \item \textbf{Depth Score} (1--5): Does the opinion provide deeper understanding than the evidence? A score of 1 indicates mere paraphrasing; 5 indicates fundamental new insights.
    
    \item \textbf{Breadth Score} (1--5): Does the opinion synthesize multiple sources? A score of 1 indicates reliance on a single source; 5 indicates comprehensive integration across references.
    
    \item \textbf{Height Score} (1--5): Does the opinion operate at a higher abstraction level? A score of 1 indicates phenomenon-level description; 5 indicates generalizable principles or hypotheses.
\end{enumerate}

\heading{Chain-of-Thought Prompting} We employ chain-of-thought prompting to elicit interpretable reasoning. The LLM first compares the opinion sentence with the retrieved evidence, then provides a rationale before assigning scores.

\heading{Composite Score} The overall insight score for a sentence is:
\begin{equation}
    S_{\text{insight}}(o_i) = \alpha \cdot S_{\text{depth}} + \beta \cdot S_{\text{breadth}} + \gamma \cdot S_{\text{height}}
\end{equation}
where $\alpha$, $\beta$, and $\gamma$ are adjustable weights (default: equal weighting).

\subsection{Stage 4: Paper-Level Report Synthesis}
The final stage aggregates sentence-level scores into a comprehensive paper-level insight report.

\heading{Score Aggregation} We compute summary statistics across all opinion sentences, including mean, median, and distribution of scores for each dimension.

\heading{Report Generation} Using the paper's introduction section and aggregated scores, we prompt an LLM to generate a natural language summary characterizing the paper's overall insightfulness. The report highlights high-scoring and low-scoring opinion sentences with explanations.

\section{Experimental Deployment}
\label{sec:experiments}
% 作者分工:夏再禹

We deployed InsightEval on a diverse corpus to validate its applicability and examine patterns in insightfulness across different writing sources.

\subsection{Dataset}
We collected 200+ academic papers comprising:
\begin{itemize}
    \item \textbf{Human-written papers} (100+): Published papers from top-tier venues (SIGIR, ACL, NeurIPS) across various research areas.
    \item \textbf{AI-generated papers} (100+): Papers generated by state-of-the-art AI writing systems, including GPT-4 and specialized academic writing tools.
\end{itemize}

\subsection{Quantitative Analysis}
Table~\ref{tab:results} presents summary statistics for the evaluated papers. Human-written papers demonstrate higher mean insight scores across all dimensions, with particularly notable differences in the height dimension.

\begin{table}[t]
    \centering
    \caption{Insight evaluation results across paper sources. Scores are on a 1--5 scale.}
    \label{tab:results}
    \begin{tabular}{lccc}
        \toprule
        \textbf{Source} & \textbf{Depth} & \textbf{Breadth} & \textbf{Height} \\
        \midrule
        Human-written & 3.42 $\pm$ 0.78 & 3.18 $\pm$ 0.85 & 3.56 $\pm$ 0.92 \\
        AI-generated & 2.87 $\pm$ 0.65 & 2.95 $\pm$ 0.71 & 2.54 $\pm$ 0.83 \\
        \bottomrule
    \end{tabular}
\end{table}

\subsection{Qualitative Observations}
Analysis of individual evaluation reports revealed several patterns:
\begin{itemize}
    \item \textbf{Human papers} tend to exhibit higher ``critical distance''---authors frequently identify limitations of cited work and propose extensions.
    \item \textbf{AI-generated papers} often achieve high breadth scores (synthesizing multiple sources) but lower depth scores (lacking novel interpretations).
    \item The height dimension shows the largest gap, suggesting that abstract reasoning and principle generalization remain challenging for AI systems.
\end{itemize}

All generated insight reports are publicly available for inspection at our project repository.

\section{Conclusion and Future Work}
\label{sec:conclusion}
% 作者分工:夏再禹

We presented InsightEval, an automated system for evaluating the insightfulness of scientific papers. By decomposing insightfulness into depth, breadth, and height dimensions and grounding evaluation in retrieved evidence from cited references, our system provides transparent and interpretable assessments. Deployment on 200+ papers demonstrates the system's applicability to both human-written and AI-generated content.

Future directions include: (1) extending evaluation beyond introduction sections to full papers; (2) incorporating user feedback to calibrate scoring models; (3) developing lightweight, fine-tuned models for cost-effective deployment; and (4) conducting human evaluation studies to validate alignment with expert judgments.

\begin{acks}
This work was supported by the National Natural Science Foundation of China.
\end{acks}


%%
%% The next two lines define the bibliography style to be used, and
%% the bibliography file.
\bibliographystyle{ACM-Reference-Format}
\bibliography{references}

\end{document}
\endinput
%%
%% End of file `paper.tex'.
