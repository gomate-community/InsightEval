\section{Experimental Deployment}
\label{sec:experiments}
% 作者分工:夏再禹

We deployed InsightEval on a diverse corpus to validate its applicability and examine patterns in insightfulness across different writing sources.

\subsection{Dataset}
We collected 200+ academic papers comprising:
\begin{itemize}
    \item \textbf{Human-written papers} (100+): Published papers from top-tier venues (SIGIR, ACL, NeurIPS) across various research areas.
    \item \textbf{AI-generated papers} (100+): Papers generated by state-of-the-art AI writing systems, including GPT-4 and specialized academic writing tools.
\end{itemize}

\subsection{Quantitative Analysis}
Table~\ref{tab:results} presents summary statistics for the evaluated papers. Human-written papers demonstrate higher mean insight scores across all dimensions, with particularly notable differences in the height dimension.

\begin{table}[t]
    \centering
    \caption{Insight evaluation results across paper sources. Scores are on a 1--5 scale.}
    \label{tab:results}
    \begin{tabular}{lccc}
        \toprule
        \textbf{Source} & \textbf{Depth} & \textbf{Breadth} & \textbf{Height} \\
        \midrule
        Human-written & 3.42 $\pm$ 0.78 & 3.18 $\pm$ 0.85 & 3.56 $\pm$ 0.92 \\
        AI-generated & 2.87 $\pm$ 0.65 & 2.95 $\pm$ 0.71 & 2.54 $\pm$ 0.83 \\
        \bottomrule
    \end{tabular}
\end{table}

\subsection{Qualitative Observations}
Analysis of individual evaluation reports revealed several patterns:
\begin{itemize}
    \item \textbf{Human papers} tend to exhibit higher ``critical distance''---authors frequently identify limitations of cited work and propose extensions.
    \item \textbf{AI-generated papers} often achieve high breadth scores (synthesizing multiple sources) but lower depth scores (lacking novel interpretations).
    \item The height dimension shows the largest gap, suggesting that abstract reasoning and principle generalization remain challenging for AI systems.
\end{itemize}

All generated insight reports are publicly available for inspection at our project repository.
